\section{Theory review}

\subsection{Self-Determination Theory(SDT)}

SDT is one of the most well grounded
\cite{motivation:handbook_self_determination} and broad theory amongst the
current theories of motivation. For SDT, the degree in which people are able to
fulfill their base psychological needs affects the pursue and attainment of
their goals critically. These needs play a vital part in SDT.

\begin{quote}
    Specifically, in SDT, three psychological needs—for competence, relatedness, and
    autonomy—are considered essential for understanding the what (i.e., content) and
    why (i.e., process) of goal pursuits. \cite{motivation:ryan_deci__what_and_why} 
\end{quote}

STD takes the perspective that people will go after purposes and goals that
permit the satisfaction of: \cite{motivation:ryan_deci__what_and_why} 

\begin{itemize}
    \item Competence -- The need to have an effect and an expertise dealing with
    your environment.
    \item Autonomy -- The need to feel in control of your own behavior.
    \item Relatedness -- The need to feel connected to your peers, to care and
    love and be cared and loved.
\end{itemize} 

\subsection{Project Based Learning(PBL)}
