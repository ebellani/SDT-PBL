\section{Theory review}

\subsection{Self-Determination Theory(SDT)}

SDT is one of the most well grounded
\cite{motivation:handbook_self_determination} and broad theory amongst the
current theories of motivation. For SDT, the degree in which people are able to
fulfill their base psychological needs affects the pursue and attainment of
their goals critically. These needs play a vital part in SDT.

\begin{quote}
    Specifically, in SDT, three psychological needs—for competence, relatedness, and
    autonomy—are considered essential for understanding the what (i.e., content) and
    why (i.e., process) of goal pursuits. \cite{motivation:ryan_deci__what_and_why} 
\end{quote}

STD takes the perspective that people will go after purposes and goals that
permit the satisfaction of: \cite{motivation:ryan_deci__what_and_why} 

\begin{itemize}
    \item Competence -- The need to have an effect and an expertise dealing with
    your environment.
    \item Autonomy -- The need to feel in control of your own behavior.
    \item Relatedness -- The need to feel connected to your peers, to care and
    love and be cared and loved.
\end{itemize} 

\subsection{Situative Learning(SL)}

SL is a framework for research on learning sciences that takes not only and
individuals but on activity systems, where the analysis encompass all the
environment where the learning takes place. In that environment there are
elements like:

\begin{quote}
    ... learners, teachers, curriculum materials, software tools, and the
    physical environment.
    \cite{education:greeno__learning_activity} 
\end{quote}


Because SL is a very abstract theory, implementations of it in more specific and
concrete situations gives rise to variations such as Constructionism:

\begin{quote}
    Papert's constructionism views learning as building relationships between old
    and new knowledge in interactions with others, while creating artifacts of
    social relevance.
    ...
    constructionism focuses on the connected nature of knowledge with its
    personal and social dimensions.
    \cite{education:yasmin__constructionism} 
\end{quote}

And Project-Based Learning(PBL):

\begin{quotation}
    Project-based learning allows students to learn by doing and applying ideas.
    Students engage in real-world activities that are similar to the activities
    that adult professionals engage in.
    
    Project-based learning is a form of situated learning \cite{education:greeno__learning_activity} 
    and it is based on the constructivist finding that students gain a deeper
    understanding of material when they actively construct their understand by
    working with and using ideas.
    \cite{education:joseph_phyllis__pbl} 
\end{quotation}

