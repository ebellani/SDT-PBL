\section{How does SL and SDT fit together?}

I plan to describe in both variations of SL, constructionism and project-based learning,
the points where it is clear that they intercalate with SDT. 

\subsection{Project-Based Learning}

I will divide the comparison following along the five key features of PBL:
driving questions, situated inquiry, collaboration learning technologies, and
artifacts.\cite{education:joseph_phyllis__pbl}

\subsubsection{Driving Questions}

The main characteristic of PBL is the driving question that guides the process
of teaching. The driving question should connect with students, allowing then to
find meaning and purpose in answering it.

When the question chosen is of good quality, it is profoundly connected to
all three pillars of SDT. That connection steams from the characteristics of a
good driving question (note the emphasized words, they are the indication of the
connection with SDT):\cite{education:joseph_phyllis__pbl}

\begin{enumerate}
\item feasibility -- Students can \emph{design} and \emph{perform}
investigations to answer the question.

\item worthwhile -- The question contains \emph{rich content} that relates to
the \emph{students communities} and to what professionals really do.

\item contextualized -- They are \emph{real world, nontrivial, and important}.

\item meaningful -- They are \emph{interesting and exciting} to learners.

\item ethical -- They do \emph{no harm to individuals, organism or the
environment}.
\end{enumerate}


