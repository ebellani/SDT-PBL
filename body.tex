\section{How does SL and SDT fit together?}

I plan to describe in one variation of SL, project-based learning, 
the points where it is clear that it intercalates with SDT. 

I will divide the comparison following along the five key features of PBL:
driving questions, situated inquiry, collaboration, learning technologies, and
artifacts.\cite{education:joseph_phyllis__pbl}

\subsection{Driving Questions}

The main characteristic of PBL is the driving question that guides the process
of teaching. The driving question should connect with students, allowing then to
find meaning and purpose in answering it.

When the question chosen is of good quality, it is profoundly connected to
all three pillars of SDT. That connection steams from the characteristics of a
good driving question (note the emphasized words, they are the indication of the
connection with SDT):\cite{education:joseph_phyllis__pbl}

\begin{enumerate}
\item feasibility -- Students can \emph{design} and \emph{perform}
investigations to answer the question.

\item worthwhile -- The question contains \emph{rich content} that relates to
the \emph{students communities} and to what professionals really do.

\item contextualized -- They are \emph{real world, nontrivial, and important}.

\item meaningful -- They are \emph{interesting and exciting} to learners.

\item ethical -- They do \emph{no harm to individuals, organism or the
environment}.
\end{enumerate}

If a task is feasible, worthwhile, contextualized, and meaningful it promotes
competence by allowing the students to be effective in a significant way. The
ethical and contextualized aspect of the question links with relatedness as the
learner affects its community in a positive fashion, driving self-respect and
pride.  Because the question is selected in a cooperative manner
\cite{education:joseph_phyllis__pbl} it attaches to autonomy.

\subsection{Situated Inquiry}

One goal of PBL is to echo the environments where the real activity takes place.
Take science for example. We can see that science classrooms differ from
laboratories, but that does not mean that it cannot consists with the
characteristics of laboratories. 

As students design their investigations, with or without the help of an
instructor, their sense of autonomy is strengthened. And when students conclude
their investigation they accomplish something that is of importance to them, and
their sense of competence is also strengthened,


\subsection{Collaborations}

PBL aims to transform the classroom in a community of learners by providing
options for all members of the learning environment to collaborate in the
investigations performed. 

\begin{quotation}
Students collaborate with others in their classroom and with their teacher to ask
questions, write explanations, form conclusions, make sense of information,
discuss data, and present findings.
...
Collaboration helps students build shared understanding of scientific ideas and
of the nature of the discipline as they engage in discourse with their
classmates and adults outside the classroom.
\cite{education:joseph_phyllis__pbl}
\end{quotation}

Because the students investigate and research in a discourse community where
they collaborate and are encouraged to flow ideas, the students feel more
connected to their peers, their professors and community. That connection feeds
their sense of relatedness.

\subsection{Learning technologies}

Technology is used to augment all other aspects of the process of learning in
the context of PBL. That process takes place in the same sense as
\cite{intelligence:engelbart__augmenting} described in his now classic work. 

Because of that characteristic, the correct use of technology can increase all
three aspects of SDT.

\subsection{Artifacts}

\begin{quote}
Learning sciences research shows that students learn more effectively when they
develop artifacts -- external representation of their constructed knowledge.
\cite{education:joseph_phyllis__pbl}
\end{quote}

By constructing and reconstructing artifacts as a result of their driving
questions, students create strong bounds between ideas that were previously
disconnected and to reflect back on their thoughts. That experience is one of
the most powerful ideas that exist \cite{education:papert_mindstorms} and
empowers learners to feel competent and to have a purpose for their efforts in
an important context.

On the other hand, the constant feedback of a teacher or a mentor is critical to
the process of construction of artifacts. Because of that a teacher must be
aware of their students progress and overall competence. Extensive feedbacks given with
respect in a proper manner increase the caring and love between all parties of
the process.

